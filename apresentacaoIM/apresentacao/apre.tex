\documentclass[t]{beamer}
\usepackage{default}
\usepackage{graphicx}
\usepackage{amsfonts}
\usepackage{amssymb}
\usepackage{amsmath}
\usepackage[brazil]{babel}
\usepackage[utf8]{inputenc}
\usepackage{lmodern}
\usetheme{Dresden}

\author{Thiago de Gouveia Nunes}
\title{Registo de Imagens\\ Demons versus TPS}
\institute{IME - USP}

\setbeamertemplate{navigation symbols}{}
\setbeamertemplate{bibliography item}{}
\setbeamertemplate{frametitle continuation}[from second]

\newcommand{\frameofframes}{/}
\newcommand{\setframeofframes}[1]{\renewcommand{\frameofframes}{#1}}

\setframeofframes{/}
\makeatletter
\setbeamertemplate{footline}
  {%
    \begin{beamercolorbox}[colsep=1.5pt]{upper separation line foot}
    \end{beamercolorbox}
    \begin{beamercolorbox}[ht=2.5ex,dp=1.125ex,%
      leftskip=.3cm,rightskip=.3cm plus1fil]{author in head/foot}%
      \leavevmode{\usebeamerfont{author in head/foot}\insertshortauthor}%
      \hfill%
      {\usebeamerfont{institute in head/foot}\usebeamercolor[fg]{institute in head/foot}\insertshortinstitute}%
    \end{beamercolorbox}%
    \begin{beamercolorbox}[ht=2.5ex,dp=1.125ex,%
      leftskip=.3cm,rightskip=.3cm plus1fil]{title in head/foot}%
      {\usebeamerfont{title in head/foot}\insertshorttitle}%
      \hfill%
      {\usebeamerfont{frame number}\usebeamercolor[fg]{frame number}\insertframenumber~\frameofframes~\inserttotalframenumber}
    \end{beamercolorbox}%
    \begin{beamercolorbox}[colsep=1.5pt]{lower separation line foot}
    \end{beamercolorbox}
  }
\makeatother

\begin{document}
\graphicspath{{images/}}
\frame{\titlepage}

\section{Revisão de Registro}
\subsection{}

\begin{frame}
    Um algoritmo de registro tenta reverter uma deformação de uma imagem a partir de dados obtidos utilizando outra imagem. \\
    Chamamos de imagem alvo a imagem que sofrerá a ação do algoritmo, enquanto a imagem referência é a imagem usada para calcular os parâmetros da transformação.
\end{frame}

\begin{frame}
    \begin{figure}[p]
        \centering
        \includegraphics[scale=0.2]{referencia.png}
        \caption{Imagem Referência}
        \label{fig:ref}
    \end{figure}
\end{frame}

\begin{frame}
    \begin{figure}[p]
        \centering
        \includegraphics[scale=0.2]{alvo.png}
        \caption{Imagem Alvo}
        \label{fig:alvo}
    \end{figure}
\end{frame}

\begin{frame}
    Um algoritmo de registro pode ser separado em:
    \begin{itemize}
        \item Transformador
        \item Medidor
        \item Otimizador
        \item Suavizador (Opcional)
        \item Pré-processador (Opcional)
    \end{itemize}
\end{frame}

\section{Demons}
\subsection{}
\begin{frame}
  O algoritmo Demons foi desenvolvido utilizando o método de fluxo óptico. Esse método é usado para encontrar um campo vetorial de deslocamento que leva
  um volume móvel até outro volume estático. \\
  A principal hipótese desse método é que as intensidades dos dois volumes são constantes.
\end{frame}

\begin{frame}
  A formula abaixo pode ser calculada utilizando a distância entre os dois hiperplanos gerados entre os volumes: \\
  \begin{equation*}
    \overrightarrow{v_i} . \overrightarrow{\triangledown}s_i = M(i) - S(i)
  \end{equation*}
\end{frame}

\begin{frame}
  A equação acima tem váriaveis demais, e não pode ser resolvida. Para resolver esse problema, o demons foi desenvolvido usando um processo
  iterativo. A cada passo um campo novo é calculado, e esse campo é aplicado a um novo volume, chamado de deformado, até que o campo se estabilize.
\end{frame}

\begin{frame}
  Algumas variantes do algoritmo:
  \begin{itemize}
    \item Cálculo do campo usando as informações dos dois volumes.
    \item Introdução de um valor para ajuste iterativo do campo.
    \item Utilização da intensidade do volume móvel ao invés do estático. 
    \item O algoritmo pode rodar em um modelo piramidal. 
  \end{itemize}
\end{frame}

\begin{frame}
    \begin{figure}[p]
        \centering
        \includegraphics[scale=0.2]{referencia.png}
        \caption{Imagem Referência}
        \label{fig:refd}
    \end{figure}
\end{frame}

\begin{frame}
    \begin{figure}[p]
        \centering
        \includegraphics[scale=0.2]{alvo.png}
        \caption{Imagem Alvo}
        \label{fig:alvod}
    \end{figure}
\end{frame}

\begin{frame}
    \begin{figure}[p]
        \centering
        \includegraphics[scale=0.2]{result.png}
        \caption{Imagem Registrada}
        \label{fig:resultd}
    \end{figure}
\end{frame}

\section{Thin-Plate Splines}
\subsection{}

\begin{frame}
    Podemos abordar o problema de outra maneira. \\
    Dado os pares de pontos $\{(x_i,y_i),(X_i,Y_i)\}$, queremos encontrar uma transformação que leve cada elemento
    ao seu par.
\end{frame}

\begin{frame}
    Essa transformação pode ser decomposta para cada uma das dimensões da imagem alvo, e assim podemos
    aproximar sua solução a construção de um híperplano em uma dimensão maior.
\end{frame}

\begin{frame}
    O Thin-Plate Splines é uma função interpoladora, dada pela equação:
    \begin{equation*}
        f(x,y) = A_1 + A_2x + A_3y + \Sigma_{i=1}^{N} F_ir_i^2 \log{r_i^2}
    \end{equation*}
    onde $r^2$ é:
    \begin{equation*}
        r^2_i = (x-x_i)^2+(y-y_i)^2+d^2
    \end{equation*}
\end{frame}

\section{Referências}
\subsection{}

\begin{frame}[allowframebreaks]
  \frametitle<presentation>{Referências}    
  \begin{thebibliography}{10}    
  \bibitem{thirion1995fast}
    Thirion, Jean-Philippe
    \newblock {\em Fast non-rigid matching of 3D medical images}.
    \newblock 1995.
  \bibitem{bookstein1989principal}
    Bookstein, Fred L.
    \newblock Principal warps: Thin-plate splines and the decomposition of deformations.
    \newblock {\em IEEE Transactions on pattern analysis and machine intelligence}, 1989.
  \end{thebibliography}
\end{frame}

\end{document}